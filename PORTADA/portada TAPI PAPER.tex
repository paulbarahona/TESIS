\documentclass[a4paper,openright,12pt]{report}
\usepackage[utf8]{inputenc}
\usepackage[spanish]{babel} % espanol
\usepackage{graphicx} % graficos
\usepackage[left=2.54cm, top=2.54cm,bottom=2.54cm, right=2.54cm]{geometry}
\begin{document}
\begin{titlepage}
\begin{center}
\vspace*{-1in}
\begin{figure}[htb]
\begin{center}

\end{center}
\end{figure}
\begin{Large}
ESCUELA POLITÉCNICA NACIONAL\\
\end{Large}
\vspace*{2.3 cm}
\begin{Large}
FACULTAD DE INGENIERÍA EN GEOLOGÍA Y PETRÓLEOS\\
\end{Large}
\vspace*{2.3 cm}
\begin{Large}
\textbf{DISTRIBUCIÓN DE LA PRODUCCIÓN DE LAS ARENAS ``U'' Y ``T'' DEL CAMPO TAPI-TETETE MEDIANTE LA APLICACIÓN DE UN MODELO NUMÉRICO.} \\
\end{Large}
\vspace*{2.2 cm}

\begin{large}
TRABAJO PREVIO A LA OBTENCIÓN DEL TÍTULO DE INGENIERO EN PETRÓLEOS \\
\end{large}
\vspace*{2 cm}
\begin{large}
CHRISTIAN PAÚL BARAHONA PAREDES \\ [0.45 cm]
paul160892@hotmail.com\\
\end{large}
\vspace*{1.4 cm}
\begin{large}
DIRECTOR:  \\
GONZALO CERÓN, Msc. \\ [0.4 cm]
gonzalo.ceron@epn.edu.ec \\[1.4 cm]
\end{large}
\begin{large}
CO-DIRECTOR:  \\
VLADIMIR CERÓN, Msc. \\ [0.4 cm]
ignacio.ceron@epn.edu.ec\\[0.4 cm]
\end{large}
\begin{large}
Quito, Junio 2018\\ [1.4 cm]
\end{large}
\end{center}
\end{titlepage}
\end{document}